\section{Opdrachten}
\subsection{2}
    Vraag 1:\\
    \begin{equation} \label{eq:a}
        C=\frac{\varepsilon_r\varepsilon_0A}{d}\Rightarrow C_{T_A}=\frac{\varepsilon_r \varepsilon_0A + \varepsilon_r \varepsilon_0}{d} \Rightarrow C = C + \frac{C}{A}\delta A \Rightarrow C=C(1+\frac{\delta A}{A}) \Rightarrow \delta C=\frac{\delta A}{A} \Rightarrow \delta C = C\frac{\delta A}{A}\Rightarrow \frac{\delta C}{C}=\frac{\delta A}{A}
    \end{equation}
    Vraag 2:\\
    \begin{equation} \label{eq:epsilonR}
        C=\frac{\varepsilon_r\varepsilon_0A}{d}\Rightarrow C_{T_A}=\frac{\varepsilon_r \varepsilon_0A + \varepsilon_0A}{d} \Rightarrow C = C + \frac{C}{\varepsilon_r}\delta\varepsilon_r \Rightarrow C=C(1+\frac{\delta \varepsilon_r}{\varepsilon}) \Rightarrow \delta C=\frac{\delta \varepsilon_r}{\varepsilon_r} \Rightarrow \delta C = C\frac{\delta \varepsilon_r}{\varepsilon_r}\Rightarrow \frac{\delta C}{C}=\frac{\delta \varepsilon_r}{\varepsilon_r}
    \end{equation}

    \noindent
    Vraag 3:\\
    $\frac{\delta C}{C}=\frac{\delta A}{A}$ is een linear verband, dit is te zien omdat er variablen zijn met een macht of een constante tot een macht en er geen variabelen in de noemer van een van de breuken staat

    \noindent
    $\frac{\delta C}{C}=\frac{\delta \varepsilon_r}{\varepsilon_r}$ is een linear verband, dit is te zien omdat er variablen zijn met een macht of een constante tot een macht en er geen variabelen in de noemer van een van de breuken staat\\

    \noindent
    Vraag 4:\\
    $\frac{\delta C}{C}=-\frac{\delta d}{d}+\frac{\delta d^2}{d^2}$ is een niet linear verband dit is te zien aan de $\frac{\delta d^2}{d^2}$ term\\

    \noindent
    Vraag 5:\\
    Van $\frac{\delta C}{C}=-\frac{\delta d}{d}+\frac{\delta d^2}{d^2}$ is $\frac{\delta d^2}{d^2}$ het niet lineare gedeelte
    en mag dus maximaal 5\% afwijken, dit komt neer op $\frac{5}{100}=\frac{\delta d^2}{d^2}\Rightarrow\frac{\sqrt{5}}{10}=\frac{\delta d}{d}\Rightarrow \frac{\sqrt{5}d}{10}=\delta d$. De maximale 
    $\delta d$ om binnen 5\% afwijking te zitten is dus te berekenen met $\delta d=\frac{\sqrt{5}d}{10}$

    \noindent
    Vraag 8:\\
    Haal uit binas gegevens gebruik binas als bron.

\subsection{3}
De opervlakte van de 1 van de platen van de schuifcondensator is 
$\pi r^2$ waarbij $r=\dots$