\section{Opdrachten}
\subsection{2}
$\bullet$ \textbf{Vraag 1}:\\
    \begin{equation} \label{eq:a}
        C=\frac{\varepsilon_r\varepsilon_0A}{d}\Rightarrow C_{T_A}=\frac{\varepsilon_r \varepsilon_0A + \varepsilon_r \varepsilon_0}{d} \Rightarrow C = C + \frac{C}{A}\delta A \Rightarrow C=C(1+\frac{\delta A}{A}) \Rightarrow \delta C=\frac{\delta A}{A} \Rightarrow \delta C = C\frac{\delta A}{A}\Rightarrow \frac{\delta C}{C}=\frac{\delta A}{A}
    \end{equation}
    $\bullet$ \textbf{Vraag 2}:\\
    \begin{equation} \label{eq:epsilonR}
        C=\frac{\varepsilon_r\varepsilon_0A}{d}\Rightarrow C_{T_A}=\frac{\varepsilon_r \varepsilon_0A + \varepsilon_0A}{d} \Rightarrow C = C + \frac{C}{\varepsilon_r}\delta\varepsilon_r \Rightarrow C=C(1+\frac{\delta \varepsilon_r}{\varepsilon}) \Rightarrow \delta C=\frac{\delta \varepsilon_r}{\varepsilon_r} \Rightarrow \delta C = C\frac{\delta \varepsilon_r}{\varepsilon_r}\Rightarrow \frac{\delta C}{C}=\frac{\delta \varepsilon_r}{\varepsilon_r}
    \end{equation}

    \noindent
    $\bullet$ \textbf{Vraag 3}:\\

    $\frac{\delta C}{C}=\frac{\delta A}{A}$ is een linear verband, dit is te zien omdat er variablen zijn met een macht of een constante tot een macht en er geen variabelen in de noemer van een van de breuken staat

    \noindent
    $\frac{\delta C}{C}=\frac{\delta \varepsilon_r}{\varepsilon_r}$ is een linear verband, dit is te zien omdat er variablen zijn met een macht of een constante tot een macht en er geen variabelen in de noemer van een van de breuken staat\\

    \noindent
    $\bullet$ \textbf{Vraag 4}:\\

    $\frac{\delta C}{C}=-\frac{\delta d}{d}+\frac{\delta d^2}{d^2}$ is een niet linear verband dit is te zien aan de $\frac{\delta d^2}{d^2}$ term\\

    \noindent
    $\bullet$ \textbf{Vraag 5}:\\

    Van $\frac{\delta C}{C}=-\frac{\delta d}{d}+\frac{\delta d^2}{d^2}$ is $\frac{\delta d^2}{d^2}$ het niet lineare gedeelte
    en mag dus maximaal 5\% afwijken, dit komt neer op $\frac{5}{100}=\frac{\delta d^2}{d^2}\Rightarrow\frac{\sqrt{5}}{10}=\frac{\delta d}{d}\Rightarrow \frac{\sqrt{5}d}{10}=\delta d$. De maximale 
    $\delta d$ om binnen 5\% afwijking te zitten is dus te berekenen met $\delta d=\frac{\sqrt{5}d}{10}$

    \noindent
    $\bullet$ \textbf{Vraag 6}:\\

    De simpelste vorm van diëlektricum dat gebruikt kan worden om (lucht)vochtigheid te meten is, naja, lucht. 
    Maar om een groot genoeg capaciteit te krijgen moeten de elektrode heel dicht op elkaar zitten. 
    En een heel groot oppervlakte hebben. Dit maakt implementatie lastig. Daarom wordt vaak voor een poreuze solide diëlektricum gekozen. 
    Water heeft bij kamertemperatuur een hoge primitiviteit $\epsilon_w=80$. 
    Hierdoor kan in een poreus materiaal met een lage primitiviteit een grote capaciteit verandering weergeven worden. 
    Des te lager de $\epsilon_w$ en des te meer poreus, des te groter de capaciteit. Een voorbeeld van een solide diëlektricum is het polymeer polytetrafluorethyleen (Teflon). 
    Teflon heeft een relatief lage primitiviteit $\epsilon_w=2.1$ .\\

    \noindent
    $\bullet$ \textbf{Vraag 7}:\\

    Haal uit binas gegevens gebruik binas als bron.

\subsection{3}
De opervlakte van de 1 van de platen van de schuifcondensator is 
$\pi r^2$ waarbij $r=\dots$