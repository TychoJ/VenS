\section{Berekening}

Bij het omslagpunt $U_{1o}$ is $U_2 = 0$.
\begin{equation}
    U_2=(U_{1o}-U_{in})\cdot\frac{R_1}{R_2+R_1}+U_{in} = 0
\end{equation}

\noindent
Dan is $U_{1o}$ dus:
\begin{equation}
    U_{1o}=U_{in}-\frac{U_{in}(R_2+R_1)}{R_1}\rightarrow U_{1o} = -U_{in}\frac{R_2}{R_1}
\end{equation}

\noindent
De formule van de integrator is:
\begin{equation} \label{eq:integrator}
    \frac{dU_{1o}}{dt}=\frac{-U_{in}}{RC}
\end{equation}

\noindent
Door dan te berekenen hoe ver $U_{1o}$ verandert bij elke $dt$ kunnen we uitrekenen wat $dt$ is. $dt$ is de halve periode tijd.
De verandering van $U_{1o}$ noemen we $dU_{1o}$. $dU_{1o}$ is de $U_{pp}$ van de driehoekgolf die uit de integrator komt.
Dat is dus 2 keer $U_{1o}$.


\begin{equation} \label{eq:dU1}
    dU_{1o}=-2U_{in}\frac{R_2}{R_1}
\end{equation}

\noindent
Door dan Formule \ref{eq:dU1} in Formule \ref{eq:integrator} in te vullen krijgen we:

\begin{equation}
    dt = \frac{2U_{in}CR_2R}{U_{in}R_1}
\end{equation}

\noindent
Dus is $C$:
\begin{equation}
    C=\frac{R_1dt}{2R_2R}
\end{equation}